\begin{enumerate}
\skiplines{1}
\item Relacion de prestamos efectuados por los prestatarios de una
determinanda comunidad.

\begin{minted}[breaklines]{sql}
WITH R(CodPrestatario, DocPrestamo) as (
    SELECT P.CodPrestatario, P.DocPrestamo
    FROM Prestamo P left outer join Amortizacion A on P.DocPrestamo = A.DocPrestamo
    GROUP BY P.CodPrestatario, P.DocPrestamo
    HAVING SUM(P.Importe) = SUM(IsNull(A.Importe, 0))
)
exec cr_CompletarPrestamos R;
go
\end{minted}

\skiplines{}
\item Relacion de prestatarios que hasta la fecha hayan efectuado mas de 5
prestamos.

\begin{minted}[breaklines]{sql}
WITH
R (CodComunidad, CodPrestatario) as (
    SELECT Pio.CodComunidad, Pio.CodPrestatario
    FROM Prestatario Pio inner join Prestamo Pmo on Pio.CodPrestatario = Pmo.CodPrestatario
)
exec cr_CompletarPrestamos R;
go
\end{minted}

\skiplines{}
\item Relacion de prestatarios morosos, es decir, aquellos que aun no han
cancelado alguna de sus deudas y ya paso la fecha de vencimiento.

\begin{minted}[breaklines]{sql}
WITH
R (CodPrestatario) as (
    SELECT Pio.CodPrestatario
    FROM Prestatario Pio inner join Prestamo Pmo on Pio.CodPrestatario = Pmo.CodPrestatario
    GROUP BY Pio.CodPrestatario
    HAVING COUNT(Pmo.DocPrestamo) > 5
)
exec cr_CompletarPrestatarios R;
go
\end{minted}

\skiplines{}
\item Relacion de las 5 comunidades que tienen el mayor numero de
prestatarios.

\begin{minted}[breaklines]{sql}
WITH
R (CodPrestatario) as (
    SELECT P.CodPrestatario
    FROM Prestamo P left outer join Amortizacion A on P.DocPrestamo = A.DocPrestamo
    GROUP BY P.DocPrestamo, P.CodPrestatario, P.FechaVencimiento, P.Importe
    HAVING SUM(IsNull(A.Importe, 0)) < P.Importe and GetDate() > P.FechaVencimiento
)
exec cr_CompletarPrestatarios R;
go
\end{minted}

\skiplines{}
\item Relacion de comunidades cuyos prestatarios que aun tienen saldos, no
hayan efectuado ninguna amortizacion en lo que va del año 2004.

\begin{minted}[breaklines]{sql}
WITH
R (CodComunidad, NroPrestatarios) as (
    SELECT TOP(5) C.CodComunidad, COUNT(P.CodPrestatario) as NroPrestatarios
    FROM Comunidad C inner join Prestatario P on C.CodComunidad = P.CodComunidad
    GROUP BY C.CodComunidad
    ORDER BY NroPrestatarios DESC
)
exec cr_CompletarComunidades R
go
\end{minted}

\skiplines{1}
\item Relacion de comunidades que no tengan prestatarios morosos.

\begin{minted}[breaklines]{sql}
WITH
R1 (CodPrestatario) as (
    SELECT P.CodPrestatario, P.DocPrestamo
    FROM Prestamo P left outer join Amortizacion A on P.DocPrestamo = A.DocPrestamo
    WHERE DATEPART(year, A.FechaCancelacion) = 2004
    GROUP BY P.DocPrestamo
    HAVING (SUM(P.Importe) > SUM(IsNull(A.Importe, 0)))
),
R2 (CodComunidad) as (
    SELECT P.CodComunidad
    FROM Prestatario P left outer join R1 on P.CodPrestatario = R1.DocPrestamo
)
exec cr_CompletarComunidades R2;
go
\end{minted}

\skiplines{1}
\item Relación de comunidades que tengan prestatarios morosos

\begin{minted}[breaklines]{sql}
WITH
R1 (CodPrestatario) as (
    SELECT P.CodPrestatario
    FROM Prestamo P left outer join Amortizacion A on P.DocPrestamo = A.DocPrestamo
    GROUP BY P.DocPrestamo, P.CodPrestatario, P.FechaVencimiento
    HAVING SUM(IsNull(A.Importe, 0)) < SUM(P.Importe) and GetDate() > P.FechaVencimiento
),
R2 (CodComunidad) as (
    SELECT P.CodComunidad
    FROM Prestatario P, R1
    WHERE P.CodPrestatario = R1.CodPrestatario
),
R3 (CodComunidad) as (
    SELECT CodComunidad
    FROM R1
    EXCEPT
    SELECT CodComunidad
    FROM R2
)
exec cr_CompletarComunidades R2;
go
\end{minted}

\end{enumerate}
