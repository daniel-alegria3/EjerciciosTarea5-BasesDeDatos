\documentclass[a4paper,openany,11pt]{article}
\usepackage[inner=1.0cm,outer=0.5cm,top=1cm,bottom=1cm]{geometry} %margen 2.54
\usepackage[utf8]{inputenc}
\usepackage[spanish]{babel}
\usepackage{amsmath}
\usepackage{csvsimple}
\usepackage{graphicx}
\usepackage{subcaption}
% \usepackage{caption}
% \captionsetup{justification=raggedright,singlelinecheck=false}
% \pagenumbering{gobble}

% \def\MINTED{} % need a way to define minted when calling tectonic
\ifdefined\MINTED
\usepackage[finalizecache]{minted}
\else
\usepackage[frozencache=true]{minted}
\fi

\usepackage{xurl}
\usepackage{hyperref}
\hypersetup{
    colorlinks = true,
    urlcolor = cyan,
    % breaklinks = true, % seems it doesnt work in tectonic?
}


\newcommand\skiplines[1]{\vspace{#1\baselineskip}}

\title{Tarea4 - Fundamentos y Diseño de Bases de Datos}
\author{Alegria Sallo Daniel Rodrigo (215270)}


\begin{document}


\section[5]{Ejercicios Complementarios}
Escribir las sentencias SELECT para obtener la siguiente informacion aplicando
los conceptos de modularidad:


\begin{enumerate}
\skiplines{1}
\item Relacion de prestamos efectuados por los prestatarios de una
determinanda comunidad.

\begin{minted}[breaklines]{sql}
WITH R(CodPrestatario, DocPrestamo) as (
    SELECT P.CodPrestatario, P.DocPrestamo
    FROM Prestamo P left outer join Amortizacion A on P.DocPrestamo = A.DocPrestamo
    GROUP BY P.CodPrestatario, P.DocPrestamo
    HAVING SUM(P.Importe) = SUM(IsNull(A.Importe, 0))
)
exec cr_CompletarPrestamos R;
go
\end{minted}

\skiplines{}
\item Relacion de prestatarios que hasta la fecha hayan efectuado mas de 5
prestamos.

\begin{minted}[breaklines]{sql}
WITH
R (CodComunidad, CodPrestatario) as (
    SELECT Pio.CodComunidad, Pio.CodPrestatario
    FROM Prestatario Pio inner join Prestamo Pmo on Pio.CodPrestatario = Pmo.CodPrestatario
)
exec cr_CompletarPrestamos R;
go
\end{minted}

\skiplines{}
\item Relacion de prestatarios morosos, es decir, aquellos que aun no han
cancelado alguna de sus deudas y ya paso la fecha de vencimiento.

\begin{minted}[breaklines]{sql}
WITH
R (CodPrestatario) as (
    SELECT Pio.CodPrestatario
    FROM Prestatario Pio inner join Prestamo Pmo on Pio.CodPrestatario = Pmo.CodPrestatario
    GROUP BY Pio.CodPrestatario
    HAVING COUNT(Pmo.DocPrestamo) > 5
)
exec cr_CompletarPrestatarios R;
go
\end{minted}

\skiplines{}
\item Relacion de las 5 comunidades que tienen el mayor numero de
prestatarios.

\begin{minted}[breaklines]{sql}
WITH
R (CodPrestatario) as (
    SELECT P.CodPrestatario
    FROM Prestamo P left outer join Amortizacion A on P.DocPrestamo = A.DocPrestamo
    GROUP BY P.DocPrestamo, P.CodPrestatario, P.FechaVencimiento, P.Importe
    HAVING SUM(IsNull(A.Importe, 0)) < P.Importe and GetDate() > P.FechaVencimiento
)
exec cr_CompletarPrestatarios R;
go
\end{minted}

\skiplines{}
\item Relacion de comunidades cuyos prestatarios que aun tienen saldos, no
hayan efectuado ninguna amortizacion en lo que va del año 2004.

\begin{minted}[breaklines]{sql}
WITH
R (CodComunidad, NroPrestatarios) as (
    SELECT TOP(5) C.CodComunidad, COUNT(P.CodPrestatario) as NroPrestatarios
    FROM Comunidad C inner join Prestatario P on C.CodComunidad = P.CodComunidad
    GROUP BY C.CodComunidad
    ORDER BY NroPrestatarios DESC
)
exec cr_CompletarComunidades R
go
\end{minted}

\skiplines{1}
\item Relacion de comunidades que no tengan prestatarios morosos.

\begin{minted}[breaklines]{sql}
WITH
R1 (CodPrestatario) as (
    SELECT P.CodPrestatario, P.DocPrestamo
    FROM Prestamo P left outer join Amortizacion A on P.DocPrestamo = A.DocPrestamo
    WHERE DATEPART(year, A.FechaCancelacion) = 2004
    GROUP BY P.DocPrestamo
    HAVING (SUM(P.Importe) > SUM(IsNull(A.Importe, 0)))
),
R2 (CodComunidad) as (
    SELECT P.CodComunidad
    FROM Prestatario P left outer join R1 on P.CodPrestatario = R1.DocPrestamo
)
exec cr_CompletarComunidades R2;
go
\end{minted}

\skiplines{1}
\item Relación de comunidades que tengan prestatarios morosos

\begin{minted}[breaklines]{sql}
WITH
R1 (CodPrestatario) as (
    SELECT P.CodPrestatario
    FROM Prestamo P left outer join Amortizacion A on P.DocPrestamo = A.DocPrestamo
    GROUP BY P.DocPrestamo, P.CodPrestatario, P.FechaVencimiento
    HAVING SUM(IsNull(A.Importe, 0)) < SUM(P.Importe) and GetDate() > P.FechaVencimiento
),
R2 (CodComunidad) as (
    SELECT P.CodComunidad
    FROM Prestatario P, R1
    WHERE P.CodPrestatario = R1.CodPrestatario
),
R3 (CodComunidad) as (
    SELECT CodComunidad
    FROM R1
    EXCEPT
    SELECT CodComunidad
    FROM R2
)
exec cr_CompletarComunidades R2;
go
\end{minted}

\end{enumerate}


\newpage
\section[6]{Investigacion Formativa}
Elaborar un resumen de las funciones internas TRANSACT SQL.
Mas informacion en \url{https://learn.microsoft.com/en-us/sql/t-sql/functions/system-functions-transact-sql}

\begin{enumerate}
    \item{\textbf{Funciones Agregadas}}
    \begin{itemize}
        \item SUM(), AVG(), COUNT(), MIN(), MAX(): Realizan cálculos en un
            conjunto de valores y devuelven un solo valor (e.g., suma,
            promedio, conteo, mínimo, máximo).
    \end{itemize}


    \item{\textbf{Funciones Analíticas}}
    \begin{itemize}
        \item ROW\_NUMBER(): Asigna un número de fila único dentro de la
            partición de un conjunto de resultados.
        \item RANK(), DENSE\_RANK(): Asignan un rango a cada fila en una
            partición, con o sin saltos.
        \item NTILE(): Divide el conjunto de resultados en un número
            especificado de grupos aproximadamente iguales.
    \end{itemize}


    \item{\textbf{Funciones de Manipulación de Bits}}
    \begin{itemize}
        \item BIT\_AND(), BIT\_OR(), BIT\_XOR(): Operan a nivel de bits en
            enteros para realizar operaciones AND, OR y XOR.
    \end{itemize}



    \item{\textbf{Funciones de Clasificación}}
    \begin{itemize}
        \item CUME\_DIST(), PERCENT\_RANK(): Calculan la posición relativa de una
            fila en una partición.
    \end{itemize}


    \item{\textbf{Funciones de Conjuntos de Filas}}
    \begin{itemize}
        \item STRING\_SPLIT(): Divide una cadena en una tabla de filas basada en un
            delimitador.
    \end{itemize}

    \item{\textbf{Funciones Escalares}}

    \begin{enumerate}
        \item Funciones de Cadena:
        \begin{itemize}
            \item LEN(), SUBSTRING(), CHARINDEX(): Operan sobre cadenas
                para obtener longitud, subcadenas, posición de subcadenas,
                etc.
            \item REPLACE(), UPPER(), LOWER(): Reemplazan subcadenas y
                cambian el caso de las letras.
        \end{itemize}

        \item Funciones de Fecha y Hora:
        \begin{itemize}
            \item GETDATE(), SYSDATETIME(): Devuelven la fecha y hora actuales.
            \item DATEADD(), DATEDIFF(): Manipulan fechas añadiendo intervalos
                o calculando diferencias.
        \end{itemize}

        \item Funciones Matemáticas:
        \begin{itemize}
            \item ABS(), CEILING(), FLOOR(): Realizan operaciones matemáticas
                básicas.
            \item ROUND(), SQRT(), POWER(): Redondean números, calculan raíces
                cuadradas y potencias.
        \end{itemize}

        \item Funciones de Conversión:
        \begin{itemize}
            \item CAST(), CONVERT(): Convierten datos de un tipo a otro.
            \item PARSE(), TRY\_PARSE(): Convierten cadenas en tipos de datos, con manejo de errores.
        \end{itemize}

        \item Funciones de Metadatos:
        \begin{itemize}
            \item OBJECT\_ID(), OBJECT\_NAME(): Recuperan el identificador y
                nombre de objetos en la base de datos.
        \end{itemize}

        \item Funciones de Seguridad:
        \begin{itemize}
            \item SUSER\_NAME(), SUSER\_SID(): Devuelven información sobre los
                inicios de sesión de SQL Server.
        \end{itemize}
    \end{enumerate}
\end{enumerate}

\end{document}

